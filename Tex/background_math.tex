

\documentclass[a5paper,12pt]{article}
\usepackage[left=0.5in,right=0.5in,top=0.5in,bottom=0.5in]{geometry}

\pagenumbering{gobble}
\usepackage{amsmath}

%%%%%%%%%%%%%%%%%%%%%%%%
%%%%%%Document
%%%%%%%%%%%%%%%%%%%%%%%%

\begin{document}

%% no field case

Formulate the problem like this:
\begin{equation}
U_{harm}=\frac{1}{2}\sum_ {\mathbf{R,R'},u,v}^{}\mathbf{u}_u(\mathbf{R})D_{uv}(\mathbf{R-R'})\mathbf{u}_v(\mathbf{R'}); \label{U}
\end{equation}

Work these out on paper:
\begin{equation}
D_{uv} (\textbf{R-R'})=\frac{\partial^2 U_{harm}}{\partial \textbf{u}_u(\textbf{R}) \partial \textbf{u}_v(\textbf{R'})}
\end{equation}

Construct this:
\begin{equation}
D(\mathbf{k})=\sum_{\textbf{R}} D(\mathbf{ R})e^{-i\mathbf{k} \cdot \mathbf{R}}
\end{equation}

Solve this:
\begin{equation}
M \omega^{2} \mathbf{\epsilon} = D(\mathbf{k})
\mathbf{\epsilon}
\end{equation}

\vspace{1cm}
\noindent The potential $U_{harm}$ can be expressed in terms of displacements of atoms from their equilibrium positions $\textbf{u}(\textbf{R})$ and force constant matrices $D(\mathbf{R})$ which give the coupling between the atoms or cells. \noindent $D(\mathbf{k})$ is the $nd\times nd$ matrix we want to construct where $n$ is the number of atoms in a unit cell and $d$ is the spatial dimension. The aim is to solve the eigenvalue problem at each $\mathbf{k}$ point. The band structure ($\omega_\sigma$ as a function of $\mathbf{k}$) is found by doing this for many $\mathbf{k}$ points. The eigenvectors $\mathbf{\epsilon}$ are indexed by $\sigma$ which runs from 1 to $nd$. They are normal modes with frequency $\omega_\sigma$. Being normal modes the motion of the system can always be expressed as a superposition of them and they are orthonormal. $M$ is the mass which will be taken as a constant here but could of course vary for different atoms types.

\newpage

%field case
Hamiltonian with a field:
\begin{equation}
H  = \frac{1}{2} (\mathbf{p}-{ A } \mathbf{u})^T (\mathbf{p}-{ A }\mathbf{u}) + \frac{1}{2} \mathbf{u}^T D(\mathbf{k}) \mathbf{u}
\end{equation}

Field term:

\begin{equation} 
A =\left( \begin{array}{cc} 0 & h \\ 
-h & 0 \end{array} \right)\;
\end{equation}



New eigenvalue problem to solve:

\begin{equation}
H_{eff}= i\left( \begin{array}{cc} -A & -D(\mathbf{k}) + A^2 \\
I_{nd} & -A \end{array} \right)
\end{equation}

\newpage

% square lattice potential

\begin{equation}
 \begin{aligned}
U_{harm} = \newline
\frac{1}{2}(\mathbf{u}(\mathbf{R}_0) - \mathbf{u}(\mathbf{R}_1))
K(\mathbf{R}_0-\mathbf{R}_1)
(\mathbf{u}(\mathbf{R}_0) - \mathbf{u}(\mathbf{R}_1))\quad + \\
\frac{1}{2}(\mathbf{u}(\mathbf{R}_0) - \mathbf{u}(\mathbf{R}_2))
K(\mathbf{R}_0-\mathbf{R}_2)
(\mathbf{u}(\mathbf{R}_0) - \mathbf{u}(\mathbf{R}_2))\quad +  \\
\frac{1}{2}(\mathbf{u}(\mathbf{R}_0) - \mathbf{u}(\mathbf{R}_3))
K(\mathbf{R}_0-\mathbf{R}_3)
(\mathbf{u}(\mathbf{R}_0) - \mathbf{u}(\mathbf{R}_3))\quad +  \\
\frac{1}{2}(\mathbf{u}(\mathbf{R}_0) - \mathbf{u}(\mathbf{R}_4))
K(\mathbf{R}_0-\mathbf{R}_4)
(\mathbf{u}(\mathbf{R}_0) - \mathbf{u}(\mathbf{R}_4))\quad +  \\
\frac{1}{2}(\mathbf{u}(\mathbf{R}_0) - \mathbf{u}(\mathbf{R}_5))
K(\mathbf{R}_0-\mathbf{R}_5)
(\mathbf{u}(\mathbf{R}_0) - \mathbf{u}(\mathbf{R}_5))\quad +  \\
\frac{1}{2}(\mathbf{u}(\mathbf{R}_0) - \mathbf{u}(\mathbf{R}_6))
K(\mathbf{R}_0-\mathbf{R}_6)
(\mathbf{u}(\mathbf{R}_0) - \mathbf{u}(\mathbf{R}_6))\quad +  \\
\frac{1}{2}(\mathbf{u}(\mathbf{R}_0) - \mathbf{u}(\mathbf{R}_7))
K(\mathbf{R}_0-\mathbf{R}_7)
(\mathbf{u}(\mathbf{R}_0) - \mathbf{u}(\mathbf{R}_7))\quad +  \\
\frac{1}{2}(\mathbf{u}(\mathbf{R}_0) - \mathbf{u}(\mathbf{R}_8))
K(\mathbf{R}_0-\mathbf{R}_8)
(\mathbf{u}(\mathbf{R}_0) - \mathbf{u}(\mathbf{R}_8))\quad \;\;\;  \\
\end{aligned} \nonumber
\end{equation}



\newpage

% kagome lattice potential

\begin{equation}
 \begin{aligned}
U_{harm} = \newline
\frac{1}{2}(\mathbf{u}_2(\mathbf{R}_0) - \mathbf{u}_1(\mathbf{R}_1))
K(\mathbf{R}_0-\mathbf{R}_1)
(\mathbf{u}_2(\mathbf{R}_0) - \mathbf{u}_1(\mathbf{R}_1))\quad + \\
\frac{1}{2}(\mathbf{u}_3(\mathbf{R}_0) - \mathbf{u}_1(\mathbf{R}_2))
K(\mathbf{R}_0-\mathbf{R}_2)
(\mathbf{u}_3(\mathbf{R}_0) - \mathbf{u}_1(\mathbf{R}_2))\quad + \\
\frac{1}{2}(\mathbf{u}_3(\mathbf{R}_0) - \mathbf{u}_2(\mathbf{R}_3))
K(\mathbf{R}_0-\mathbf{R}_3)
(\mathbf{u}_3(\mathbf{R}_0) - \mathbf{u}_2(\mathbf{R}_3))\quad + \\
\frac{1}{2}(\mathbf{u}_1(\mathbf{R}_0) - \mathbf{u}_2(\mathbf{R}_4))
K(\mathbf{R}_0-\mathbf{R}_4)
(\mathbf{u}_1(\mathbf{R}_0) - \mathbf{u}_2(\mathbf{R}_4))\quad + \\
\frac{1}{2}(\mathbf{u}_1(\mathbf{R}_0) - \mathbf{u}_3(\mathbf{R}_5))
K(\mathbf{R}_0-\mathbf{R}_5)
(\mathbf{u}_1(\mathbf{R}_0) - \mathbf{u}_3(\mathbf{R}_5))\quad + \\
\frac{1}{2}(\mathbf{u}_2(\mathbf{R}_0) - \mathbf{u}_3(\mathbf{R}_6))
K(\mathbf{R}_0-\mathbf{R}_6)
(\mathbf{u}_2(\mathbf{R}_0) - \mathbf{u}_3(\mathbf{R}_6))\quad + \\
\\
\frac{1}{2}(\mathbf{u}_1(\mathbf{R}_0) - \mathbf{u}_2(\mathbf{R}_0))
K(\mathbf{R}_{01}-\mathbf{R}_{02})
(\mathbf{u}_1(\mathbf{R}_0) - \mathbf{u}_2(\mathbf{R}_0))\; + \\
\frac{1}{2}(\mathbf{u}_2(\mathbf{R}_0) - \mathbf{u}_3(\mathbf{R}_0))
K(\mathbf{R}_{02}-\mathbf{R}_{03})
(\mathbf{u}_2(\mathbf{R}_0) - \mathbf{u}_3(\mathbf{R}_0))\; + \\
\frac{1}{2}(\mathbf{u}_3(\mathbf{R}_0) - \mathbf{u}_1(\mathbf{R}_0))
K(\mathbf{R}_{03}-\mathbf{R}_{01})
(\mathbf{u}_3(\mathbf{R}_0) - \mathbf{u}_1(\mathbf{R}_0))\;\;\;\ \\
\end{aligned} \nonumber
\end{equation}

\end{document}


